% % % % % % % % % % % % % % % % % % % % % % % % % % % % % % % % % % % % % % % % 
% Nombre: 		LaTeX Parte 2.tex
% Fecha:  		12-01-2014
% Autor:  		jm@taquinux.com
% % % % % % % % % % % % % % % % % % % % % % % % % % % % % % % % % % % % % % % % 

\Chapter%
         % Titulo como aparece al comenzar el articulo
         {\LaTeX{} Parte 2}%
         % Titulo como aparece en el índice
         {\LaTeX{} Parte 2 Mas detalles}%
         % Resumen que aparece debajo del titulo
         {Continuación del articulo anterior para mostrar algunas funcionalidades
         adicionales}%
         % Autor
         {José Miguel Jáuregui García}%
         % Imagen que aparece después del resumen 
         %{Tux.png}

% % % Definición de uso de dos columnas.
\begin{multicols}{2}

\section*{Artículo a dos columnas}
El formato de dos columnas es más fácil de leer para hacer que el artículo aparezca
con este formato después de haber definido la cabecera del artículo debe de ir
rodeado por los siguientes comandos:
\Code
   {LaTeX}
   {
   \\*
   $\setminus$begin$\{$multicols$\}\{$2$\}$ \\
   <contenido del articulo> \\
   $\setminus$end$\{$multicols$\}\{$2$\}$
   }

\section*{Saltos de línea y espacios en blanco}
\LaTeX{} nos brinda la facilidad de que podemos poner tantos espacios entre las palabras
del texto como queramos, poner lineas entre los párrafos y al compilarse elimina esos
espacios en blanco para tener un formato uniforme, pero en ocasiones es necesario incluir
un salto de línea o espacios en blanco entre palabras eso se puede hacer con:
\Code
   {LaTeX}
   {
   \\*
   $\setminus\setminus$ Para forzar un salto de línea\\
   $\setminus$: Para poner un espacio en \: blanco \\
   }

\section*{Firma de documento}
Para agregar una firma hay que editar las líneas que aparecen casi al finalizar el artículo,
se pueden agregar tantas líneas como creamos necesarias solo hay que agregar los saltos
de línea necesarios para que aparezca con el formato correcto.
\Code
   {LaTeX}
   {
   \\*
	$\setminus$begin$\{$flushright$\}$ \\
	Nombre \\
	correoelectronico \\
	$\setminus$end$\{$flushright$\}$
   }


\section*{Codigo}
Para los bloques de código los saltos de línea son manuales y se hacen agregando
doble diagonal al final de la línea hay un máximo de 37 caracteres por articulo
de dos columnas y 75 para articulo de columna sencilla.
% % % INICIA CUADRO DE CODIGO
\Code
   {Script}
   {
   \\*
   0000000001111111111222222222233333333\\
   1234567890123456789012345678901234567\\
   }
% % % TERMINA CUADRO DE CODIGO

% % % INICIA FIRMA DE ARTICULO
\begin{flushright}
José Miguel Jáuregui García \\
taquinux@gmail.com
\end{flushright}
% % % FINALIZA FIRMA DE ARTICULO


\end{multicols}
% % % Fin de uso de dos columnas